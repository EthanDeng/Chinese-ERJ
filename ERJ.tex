\documentclass[10pt]{ERJ}

\title{ERJ:《经济研究》杂志 \LaTeX{} 模板}
\author{邓东升}
\date{}
\usepackage{abstract}
\renewcommand{\abstractname}{}    % clear the title
\renewcommand{\absnamepos}{empty} % originally center

\usepackage[backend=biber,style=erj]{biblatex}
\addbibresource{erjref.bib}
\usepackage{indentfirst}
\setlength{\parindent}{2em}
\renewcommand{\abstracttextfont}{\normalsize}
\begin{document}
\maketitle

\begin{abstract}
\par 内容提要:加工贸易转型升级是中国转变发展方式和改革开放战略的重要内容。本文以“大学扩招”政策的实施作为准自然实验,采用倍差法系统地评估了人力资本对中国加工贸易企业升级的影响及其作用机制。本文发现,人力资本扩张显著提高了加工贸易企业的出口技术复杂度,有利于促进加工贸易企业升级。渠道检验表明,人力资本扩张不仅促使了加工贸易企业加大研发投入和在职培训的力度,而且还促进了加工贸易企业进口使用更多种类和更高质量的中间投入品,同时还激励了加工贸易企业增加固定资产投资,这些因素共同推动了加工贸易企业升级。异质性分析发现,人力资本扩张对融资约束程度低、管理效率高、资本密集型以及外资型加工贸易企业升级的促进效应更大。此外,本文从更多的维度研究了人力资本与加工贸易升级的关系,发现人力资本扩张促进了加工贸易企业组织方式从来料加工向进料加工的转变,提高了加工贸易企业的出口国内附加值率,同时还提升了加工贸易企业的生产效率与自主创新能力。\par
关键词:人力资本 加工贸易升级 高等教育改革
\end{abstract}


\section{经济研究}
中国加工贸易自改革开放以来所取得的发展成就有目共睹,加工进出口额从 1983 年的 42. 2长速度。中国加工贸易的迅速发展对于促进经济增长、扩大就业、外贸增长、利用外资、促进企业国际化发展、推动产业结构调整、保持外汇平衡方面产生了重要的作用(隆国强和张丽平,2012;Yu)加工贸易转型升级是中国转变发展方式和改革开放战略的重要内容~\cite{约翰1978--}。本文以“大学扩招”政策的实施作为准自然实验,采用倍差法系统地评估了人力资本对中国加工贸易企业升级的影响及其作用机制。本文发现,人力资本扩张显著提高了加工贸易企业的出口技术复杂度,有利于促进加工贸易企业升级。渠道检验表明,人力资本扩张不仅促使了加工贸易企业加大研发投入和在职培训的力度,而且还促进了加工贸易企业进口使用更多种类和更高质量的中间投入品,同时还激励了加工贸易企业增加固定资产投资,这些因素共同推动了加工贸易企业升级。异质性分析发现,人力资本扩张对融资约束程度低、管理效率高、资本密集型以及外资型加工贸易企业升级的促进效应更大~\cite{John1978--}。此外,本文从更多的维度研究了人力资本与加工贸易升级的关系,发现人力资本扩张促进了加工贸易企业组织方式从来料加工向进料加工的转变~\cite{李四1991--},提高了加工贸易企业的出口国内附加值率,同时还提升了加工贸易企业的生产效率与自主创新能力~\cite{Carlson2000}。



 \printbibliography
\end{document}
