\documentclass{ERJ}
\title{\huge 中国用能权交易可以获得经济红利\\与节能减排的双赢吗?\thanks{张宁,暨南大学经济学院经济系,邮政编码:510632,电子信箱:zhangn@ jnu. edu. cn;张维洁,暨南大学经济学院经济系博
士研究生。本研究国家自然科学优秀青年基金项目(71822402)、国家自然科学基金重大研究计划(91746112)和国家社科基金重
大项目 (15ZDA054)的资助。作者感谢匿名审稿人的意见,文责自负。}}
\author{\Large 张 \quad 宁 \; 张维洁}
\date{}

\renewcommand*{\bibfont}{\scriptsize}
\setlength\bibitemsep{0.2\baselineskip}
\renewcommand{\refname}{\small  参考文献}

\renewcommand{\footnotesize}{\scriptsize}

\usepackage{titling}

\setlength{\droptitle}{-4.5em}
\renewcommand\maketitlehookb{\vspace{-1.2ex}}
\begin{document}

\maketitle
\thispagestyle{fancy}
\vskip -5em

\begin{abstract}
\par 内容提要:   命令控制和市场机制的环境规制政策,孰能促进经济与环境的双赢一直是
学界争论的话题。新古典经济理论认为,基于市场的政策能以最低的成本达到环境治理
目标,从而实现经济与环境的双赢发展。2016 年中国政府开创性地提出了用能权交易政
策,这项政策的经济与环境效应还尚未有文献研究。本文利用 38 个二位数工业分行业数
据,分别构造命令控制型和用能权交易型的非参数优化模型,模拟了 2006—2020 年两种
政策下中国工业行业的经济潜力和节能潜力,试图弥补能源经济学中关于能源许可证交
易政策研究的不足。本文的结论是,与命令控制政策相比,用能权交易政策均会带来较高
的平均经济潜力和节能潜力。但是,具体到各行业,由于市场主体逐利的本质,部分行业
的平均节能潜力会被挤出。因此,在实施用能权交易政策时,必须坚持市场交易为主,政
府调控为辅,根据各行业不同的经济潜力和节能潜力设计各行业初始能源配额,利用用能
权交易来实现中国工业经济增长和节能减排的双赢发展。\par
关键词: 工业行业\quad 命令控制\quad 用能权交易\quad\renewcommand\maketitlehookc{\vspace{-10ex}}市场机制
\end{abstract}
\vskip 2em
\section{引   言}
 
\subsection{单文献引用}
\noindent 中国经济已经由高速增长阶段转向高质量发展阶段 \cite{cn1} 说了啥\\
中国经济已经由高速增长阶段转向高质量发展阶段 \cite{cn2} 。说了啥\\
中国经济已经由高速增长阶段转向高质量发展阶段 \cite{cn3} 。说了啥\\
中国经济已经由高速增长阶段转向高质量发展阶段 \cite{en1} 。说了啥\\
中国经济已经由高速增长阶段转向高质量发展阶段 \cite{en2} 。说了啥\\
中国经济已经由高速增长阶段转向高质量发展阶段 \cite{en3} 。定性\\
中国经济已经由高速增长阶段转向高质量发展阶段 \cite{en3} 。\\
\subsection{单文献引用}
中国经济已经由高速增长阶段转向高质量发展阶段,\noindent \textcite{cn1} 说了啥\\
中国经济已经由高速增长阶段转向高质量发展阶段,\textcite{cn2} 说了啥中国经济已经由高\\
中国经济已经由高速增长阶段转向高质量发展阶段,\textcite{cn3} 说了啥\\
中国经济已经由高速增长阶段转向高质量发展阶段,\textcite{en1} 说了啥\\
中国经济已经由高速增长阶段转向高质量发展阶段,\textcite{en2} 说了啥\\
中国经济已经由高速增长阶段转向高质量发展阶段,\textcite{en3} 说了啥\\
\newpage

\subsection{多文献引用}
\noindent\cite{cn1,cn2}\\
中国经济已经由高速增长阶段转向高质量发展阶段\cite{cn1,cn3}\\
中国经济已经由高速增长阶段转向高质量发展阶段 \cite{cn2,cn3}\\
中国经济已经由高速增长阶段转向高质量发展阶段\cite{cn1,cn2,cn3}\\
中国经济已经由高速增长阶段转向高质量发展阶段\cite{en1,en2}\\
中国经济已经由高速增长阶段转向高质量发展阶段\cite{en1,en3}\\
中国经济已经由高速增长阶段转向高质量发展阶段\cite{en2,en3}\\
中国经济已经由高速增长阶段转向高质量发展阶段\cite{en1,en2,en3}\\

\subsection{多文献文本引用}
\noindent\textcite{cn1,cn2} 中国经济已经由高速增长阶段转向高质量发展阶段\\
长阶段转向高质量发展阶,\textcite{cn1,cn3} 中国经济已经由高速增长阶段转向高质量发展阶段\\
\textcite{cn2,cn3} 中国经济已经由高速增长阶段转向高质量发展阶段\\
\textcite{cn1,cn2,cn3} 中国经济已经由高速增长阶段转向高质量发展阶段\\
\textcite{en1,en2} 中国经济已经由高速增长阶段转向高质量发展阶段\\
\textcite{en1,en3} 中国经济已经由高速增长阶段转向高质量发展阶段\\
\textcite{en2,en3} 中国经济已经由高速增长阶段转向高质量发展阶段\\
长阶段转向高质量发展阶~\textcite{en1,en2,en3} 中国经济已经由高速增长阶段转向高质量发展阶段\\

中国经济已经由高速增长阶段转向高质量发展阶段,传统的以能源消耗为代价的增长模式对
于经济的拉动已显乏力。中国工业是能源的消耗主体,2005 年中国工业的能源消耗占全国
64. 6\% ,2014 年这一比例增加至 69. 4\% ,但中国工业增加值占 GDP 的比重却在逐渐下降,以六大高能耗行业为例,由 2005 年的 14. 3\% 下降至 2014 年的 7. 3\% ,可见这种能源驱动型的增长方式已面临瓶颈。在巨大的能源消费中,煤炭是主要的消费来源(林伯强和李江龙,2014)。2014 年中国工业煤炭消费量为 39. 05 亿吨,占全国煤炭消费总量的 94. 87\% 。这种化石能源主导的能源消费结构不但消耗大量资源,还造成大气污染、雾霾、气候变化等环境问题,严重影响了中国经济的可持续发展和人民的生活质量。为此,党的十九大把绿色发展列入“五位一体”总布局中,新的发展理念下,高能耗增长模式已经难以为继,选择合理有效的环境政策,在促进经济增长的同时降低能耗,实现经济、能源与环境的协调发展已经成为当前中国发展亟待解决的重大问题(邵帅等,2013)。

环境管制政策可分为命令控制型环境政策和基于市场的环境政策。长期以来,中国政府主要采取命令控制型环境政策来配置资源、治理环境,比如通过法律和行政手段为地区和行业企业制定环境标准和目标,对违反规定的企业叫停和处罚等。自“十五”以来,基于市场的环境政策逐步出现,比如二氧化硫排污权交易、碳排放权交易等,这些排放许可证交易政策主要借鉴发达国家,逐渐对中国的节能减排发挥日益重要的作用。但是,可以发现基于市场的交易许可证政策基本上都是针对污染排放,即属于末端治理。为了实现节能减排,中国政府开始注
重源头投入治理,“十三五”时期提出了控制能源消费总量和能源强度的“双控”目标。为了实
现该目标,中国政府开创性地提出了“用能权交易制度”,在浙江、福建、河南和四川等省份开展用
能权有偿使用和交易试点。① 用能权交易,是指在用能总量控制的前提下,参与主体对依法取得的
用能总量指标进行交易的行为。
设定约束性节能目标和实施用能权交易是中国控制工业能耗的两大政策,分别属于命令控
制型环境政策和基于市场的环境政策。哪种政策能实现经济增长和节能减排的双赢发展,一直
是存在争议的话题。Porter(1991) 和 Porter et al. (1995) 认为政府实施严格的环境管制可以促使
企业创新,创新带来的利润可以补偿环境保护的额外成本,从而实现经济增长和节能减排的双赢
发展。但是,Jaffe et al. (2003)认为命令控制政策具有很高的成本,对环境技术标准的设定还会
阻碍企业的技术进步。基于市场的环境政策不但可以在最小成本下达到环境治理目标,还具有
技术创新的持续激励,可以带来经济增长与节能减排的双赢发展。理论上来讲,在完全信息的环
境下,设定节能目标的命令控制政策与基于市场的许可证交易政策的效果是等价的,但实际中由
于信息不对称,且不同生产者之间的发展水平、技术水平、资源禀赋和能耗水平差距较大,命令控
制政策没有充分考虑生产者服从管制的成本,可能会损害部分经济主体的利益,从而造成一定的
经济损失。而用能权交易这种基于市场的政策则可以通过市场手段解决资源配置无效率的问
题,激励经济主体改进生产技术,在释放节能潜力的同时创造经济收益。用能权交易政策作为一
项创新性的投入许可证交易政策,对其经济和环境效应的研究,不但对实现中国可持续发展和政
策推广具有重要的现实意义,还可以弥补能源经济学关于这一方面研究的不足。基于此,本文将
研究对象聚焦在 38 个工业分行业层面,从整体层面和分行业层面对用能权交易政策下中国工业
的经济潜力和节能潜力进行探讨, ② 并试图回答以下两个问题:(1) 中国工业用能权交易的经济
潜力和节能潜力存在吗? 有多大? (2)与严格的命令控制政策相比,基于市场的用能权交易政策
对协调经济发展和环境保护是否更加有效?
\section{文献综述}
(一)环境政策的讨论

环境资源是一种不具有排他性但具有竞争性的公共产品,具有严重的外部性,必须依靠环境管
制来解决。常见的环境管制政策分为命令控制型环境政策和基于市场的政策。传统的政府干预经
济理论认为,由于外部性的内部化无法依靠市场机制实现,政府拥有完全信息,因此命令控制政策
在解决环境外部性方面是有效的。基于这一理论,Porter et al. (1995)肯定了政府管制对解决环境
问题的作用,并且进一步提出严格的管制会诱发企业创新和技术进步,实现经济与环境的双赢,即
“波特双赢假说”。Boyd et al. (1999)和 Boyd et al. (2000) 对造纸厂和玻璃厂的实证研究发现,政
府严格的环境管制在没有降低企业生产率的情况下使污染减少了 2\% —8\% ,支持了“波特双赢假
说”。Simon(1976)、Nelson et al. (1982)、Greenstone & Hanna(2014)、Shapiro & Walker(2015)均做
了类似的研究,不同程度地证实了政府严格管制对环境的改善作用。

然而,新古典经济理论认为,政府不可能拥有完全信息,而且命令控制型的环境政策通常倾向。

\vspace*{0.8\baselineskip}
\begin{flushleft}
\small 参考文献
\vspace*{-0.3\baselineskip}
\end{flushleft}
\printbibliography[heading=none]

\end{document}
